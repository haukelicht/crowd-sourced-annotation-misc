\documentclass[]{article}
\usepackage{lmodern}
\usepackage{amssymb,amsmath}
\usepackage{ifxetex,ifluatex}
\usepackage{fixltx2e} % provides \textsubscript
\ifnum 0\ifxetex 1\fi\ifluatex 1\fi=0 % if pdftex
  \usepackage[T1]{fontenc}
  \usepackage[utf8]{inputenc}
\else % if luatex or xelatex
  \ifxetex
    \usepackage{mathspec}
  \else
    \usepackage{fontspec}
  \fi
  \defaultfontfeatures{Ligatures=TeX,Scale=MatchLowercase}
\fi
% use upquote if available, for straight quotes in verbatim environments
\IfFileExists{upquote.sty}{\usepackage{upquote}}{}
% use microtype if available
\IfFileExists{microtype.sty}{%
\usepackage{microtype}
\UseMicrotypeSet[protrusion]{basicmath} % disable protrusion for tt fonts
}{}
\usepackage[margin=1in]{geometry}
\usepackage{hyperref}
\hypersetup{unicode=true,
            pdftitle={Measuring populism using nosiy human codings of textual data},
            pdfauthor={Hauke Licht},
            pdfborder={0 0 0},
            breaklinks=true}
\urlstyle{same}  % don't use monospace font for urls
\usepackage{longtable,booktabs}
\usepackage{graphicx,grffile}
\makeatletter
\def\maxwidth{\ifdim\Gin@nat@width>\linewidth\linewidth\else\Gin@nat@width\fi}
\def\maxheight{\ifdim\Gin@nat@height>\textheight\textheight\else\Gin@nat@height\fi}
\makeatother
% Scale images if necessary, so that they will not overflow the page
% margins by default, and it is still possible to overwrite the defaults
% using explicit options in \includegraphics[width, height, ...]{}
\setkeys{Gin}{width=\maxwidth,height=\maxheight,keepaspectratio}
\IfFileExists{parskip.sty}{%
\usepackage{parskip}
}{% else
\setlength{\parindent}{0pt}
\setlength{\parskip}{6pt plus 2pt minus 1pt}
}
\setlength{\emergencystretch}{3em}  % prevent overfull lines
\providecommand{\tightlist}{%
  \setlength{\itemsep}{0pt}\setlength{\parskip}{0pt}}
\setcounter{secnumdepth}{0}
% Redefines (sub)paragraphs to behave more like sections
\ifx\paragraph\undefined\else
\let\oldparagraph\paragraph
\renewcommand{\paragraph}[1]{\oldparagraph{#1}\mbox{}}
\fi
\ifx\subparagraph\undefined\else
\let\oldsubparagraph\subparagraph
\renewcommand{\subparagraph}[1]{\oldsubparagraph{#1}\mbox{}}
\fi

%%% Use protect on footnotes to avoid problems with footnotes in titles
\let\rmarkdownfootnote\footnote%
\def\footnote{\protect\rmarkdownfootnote}

%%% Change title format to be more compact
\usepackage{titling}

% Create subtitle command for use in maketitle
\newcommand{\subtitle}[1]{
  \posttitle{
    \begin{center}\large#1\end{center}
    }
}

\setlength{\droptitle}{-2em}

  \title{Measuring populism using nosiy human codings of textual data}
    \pretitle{\vspace{\droptitle}\centering\huge}
  \posttitle{\par}
    \author{Hauke Licht}
    \preauthor{\centering\large\emph}
  \postauthor{\par}
      \predate{\centering\large\emph}
  \postdate{\par}
    \date{2019-04-08}


\begin{document}
\maketitle

\begin{verbatim}
## NULL
\end{verbatim}

\hypertarget{introduction}{%
\subsection{Introduction}\label{introduction}}

With the increased scholarly interest in populism that political science
has witnessed in the last two decades, the measurement of populism has
presented itself as a major challenge to the discipline. Different
instruments have been proposed to elicit measurements of populism (for
review articles see Pauwels 2017; Bergman 2018; Kirk A. Hawkins et al.
2018), ranging from

\begin{itemize}
\tightlist
\item
  actor-based classifications by experts (e.g., Polk et al. 2017)
\item
  over content analytical approaches relying on human judgment (e.g.,
  Jagers and Walgrave 2007; Hawkins 2009; Rooduijn and Pauwels 2011;
  Rooduijn, de Lange, and van der Brug 2014; Ernst et al. 2017;
  Aslanidis 2018; March 2018) or computerized dictionary methods (e.g.,
  Pauwels 2011; Rooduijn and Pauwels 2011; Oliver and Rahn 2016)
\item
  to approaches based on machine learning approaches that learn to
  predict populist instances from human annotated training data (Hua,
  Abou-Chadi, and Barberá 2018; Kirk A Hawkins and Silva 2018) or use
  word-embedding techniques ({\textbf{???}}; Dai 2019).
\end{itemize}

This research note is addressing a challenge to the content-analytical
measurement of populism that has hitherto received littel to no
attention. Classical content-analytical measurement instruments present
human coders with instances of text, which they then are asked to judge
in accordance with a coding scheme (Krippendorff 2004). While
researchers usually translate their theoretical concept of populism into
a categorical coding scheme with the goal to illicit correct
measurements, measurement ultimately relies on the judgement of human
coders. Human coders---irrespectige of their domain-specific expertise
and their level of prior training---are thus considered `noisy labelers'
in the statistical literature (Dawid and Skene 1979; Passonneau and
Carpenter 2014; Guan et al. 2017). Coders varying abilities to correctly
classify or rate instances of text thus generally constitutes one source
of error in content-analytical measurements of populism that is not
eliminable by design.

Depending on the degree of human coders' falibility and the aggregation
method used to obtain measurements at the level of instances, such
coding error may impair measurement quality. This applies most
specifically to conventional, non-parametric aggregation methods such as
majority voting that due not account for the possibility of
agreement-in-error (i.e., the majority of coders agrees on the false
judgement, see Passonneau and Carpenter 2014). The goal of my analysis
is thus to assess the variability and the degree of imperfection in
human coders abilities to classify instances along the dimensions of a
categorical coding scheme design to measure populism in textual data.

To do so I fit Bayesian annotation models to the human codings Hua,
Abou-Chadi, and Barberá (2018) have collect on the crowd-sourcing
platform \emph{CrowdFlower}. Specifically, I fit the Beta-Binomial by
Annotator (BBA) model proposed by Carpenter (2008) that allows to
estimate the positive class's prevalence, items' class membership, as
well as coders' individual specificities and sensitivities from the
codings data. Assessing coders' abilities in this `crowd' of untrained
human coders allows to determine a lower benchmark on the aggregate
measurement quality their judgements yield. The specific case I select
for analysis thus constitutes a hard test for my argument: If
model-based estimates of human coders' abilities indicate
close-to-perfect coder abilities, we would be less concerned that
aggreement-in-error causes false classifications of instance. Likewise,
if the model-based aggregation of noisy labels into instance-level
measruements does not yield substantively different classifications that
do non-parametric aggregation methods, the added value of obtaining
estimates of coder abilities seems unjustified.

The remainder of this reasearch note is structured as follows: First, I
introduce the measurements instrument used by Hua, Abou-Chadi, and
Barberá (2018) to obtain their crowd codings and describe the data. I
then introduce the BBA model, discuss its notation, and show results
from a simulation study that demonstrate that its implementation in JAGS
({\textbf{???}}) allows to recover simulated parameter values.

\hypertarget{data-and-empirical-strategy}{%
\subsection{Data and empirical
strategy}\label{data-and-empirical-strategy}}

\hypertarget{crowd-sourced-codings-of-populism-in-textual-data}{%
\subsubsection{Crowd-sourced codings of populism in textual
data}\label{crowd-sourced-codings-of-populism-in-textual-data}}

Hua et al.~recruited crowd workers on the crowd-sourcing platform
\emph{CrowdFlower} to code social media posts created by a selected
number of accounts of Wesrtern European parties and their leaders
according to the following coding scheme:

\begin{enumerate}
\def\labelenumi{\arabic{enumi}.}
\tightlist
\item
  Filter questions:

  \begin{enumerate}
  \def\labelenumii{\arabic{enumii}.}
  \tightlist
  \item
    This post has no text or its content is impossible to understand.
  \item
    I understand the message of this social media post.
  \end{enumerate}
\item
  \emph{Anti-elitism}: Does this tweet/post criticize or mention in a
  negative way the elites?
\item
  \emph{People-centrism}: Does this tweet/post mention in a positive way
  or even praise the people (citizens of the country, the working class,
  the native \ldots) or the nation?
\item
  \emph{Exclusionism}: Does this tweet/post criticize minorities or
  specific groups of people (muslims, jews, LGBT people, poor people
  \ldots)?
\end{enumerate}

Coders were asked to answer Yes or No to all questions; all judgements
obtained are thus binary. If a coder answered question 1.1 affirmatively
for a post, she was asked to skip the given post and proceed with
judging the next one. If a coder answered question 1.2 affirmatively for
a post, she was asked to proceed with answering questions 2--4 and then
proceed with judging the next post.

Of particular interst are the judgments Hua et al.~collected in an
effort to assertain the reliability of their crowd-source measurement
instrument. Interested in computing inter-coder agreement metrics, Hua
et al.~crowd-sourced judgments from 1 different coders for a set of 1
different social media posts (items). As the following table shows, each
item was coded between 1 and four times.

\begin{longtable}[]{@{}rrr@{}}
\toprule
No.~Judgments & No.~Coders & \(N\)\tabularnewline
\midrule
\endhead
1 & 1 & 507\tabularnewline
2 & 2 & 89\tabularnewline
3 & 3 & 902\tabularnewline
4 & 4 & 2\tabularnewline
\bottomrule
\end{longtable}

Keeping all judgments that passed the first two filter
questions,\footnote{These were all items for which the meta variable
  \texttt{fitler} had the value `ok'.} I was able to retain a total 1489
items from the original validation data.

\hypertarget{the-bayesian-beta-binomial-by-annotator-model}{%
\subsubsection{The Bayesian Beta-Binomial by Annotator
model}\label{the-bayesian-beta-binomial-by-annotator-model}}

I assume that the anti-elitism, people-centrism, and exclusionism are
latent binary features of political posts, and hence crowd coders act as
human content analysts whose judgments I want to aggregate at the
item-level to estimate whether a given item belongs to either of these
three categories.

For each dimension, the setup can be described as a four-tuple
\(\langle\mathcal{I}, \mathcal{J}, \mathcal{K}, \mathcal{Y}\rangle\),
where

\begin{itemize}
\tightlist
\item
  \(\mathcal{I}\) is the set of \emph{items} \(i \in 1,\ \ldots ,\ n\)
  distributed for crowd coding,
\item
  \(\mathcal{J}\) is the set of \emph{coders} \(j \in 1,\ \ldots ,\ m\),
\item
  \(\mathcal{K}\) is the set of \emph{classes} \(k \in \{0, 1\}\)
  defined by the categorical coding scheme used during crowd-coding, and
\item
  \(\mathcal{Y}\) is the set of \emph{judgments} (or codings)
  \(y_{i,j} \in \{0, 1\}\) recorded for item \(i\) by coder \(j\).
\end{itemize}

In Hua et al.'s validation data, it is provided that \(\mathcal{Y}\)
contains at least one judgment per item, that is,
\(|\mathcal{Y}_i| \geq 1\ \forall\ i \in \mathcal{I}\), and that
\(|\mathcal{Y}_i| \geq 2\ \forall\ i \in \mathcal{I}' \subset \mathcal{I}\).
Moreover, judgments for each item in \(\mathcal{I}'\) are generated by
different coders (i.e., we have repeated annotation at the statement
level, but not at the judgment-item level).

Importantly, while coders' judgments of items are observed, true class
labels \(l_i \in \mathcal{K}\) are unknown \emph{a priori} for all
\(i = 1,\ \ldots,\ I\). In this setup, a classification of items into
classes obtained from a set of judgments
\(\rho(\mathcal{Y}) \Rightarrow \mathcal{L}\) is called a \emph{`ground
truth'} or \emph{labeling}. I obtain these ground truth estimates by
fitting the following model to the judgement data:

\[
\begin{align*}
c_i &\sim\ \mbox{Bernoulli}(\pi)\\
\theta_{0j} &\sim\ \mbox{Beta}(\alpha_0 , \beta_0)\\
\theta_{1j} &\sim\ \mbox{Beta}(\alpha_1 , \beta_1)\\
y_{ij} &\sim\ \mbox{Bernoulli}(c_i\theta_{1j} + (1 - c_i)(1  - \theta_{0j}))\\
{}&{}\\
\pi &\sim\ \mbox{Beta}(1,1)\\
\alpha_0/(\alpha_0 + \beta_0) &\sim\ \mbox{Beta}(1,1)\\  
\alpha_0+\beta_0 &\sim\ \mbox{Pareto}(1.5)\\
\alpha_1/(\alpha_1 + \beta_1) &\sim\ \mbox{Beta}(1,1)\\ 
\alpha_1+\beta_1  &\sim\ \mbox{Pareto}(1.5)
\end{align*}
\] where

\begin{itemize}
\tightlist
\item
  \(c_i\) is the `true' (unobserved) class of statement \(i\),
\item
  \(\pi\) is the `true' prevalence of the positive class,
\item
  \(\theta_{0,j}\) is coder \(j\)'s specificity (true-negative rate),
  and
\item
  \(\theta_{1,j}\) is her sensitivity (true-positive rate).
\end{itemize}

Carpenter (2008) refers to this model as the Beta-Binomial by Annotator
(BBA) model. This name is due to its property that, given a conjugate
beta prior, the posterior densities of items' class membership follow a
beta-binomial distribution. All priors are choosen to be uninformative,
as we have no prior knowledge in about coders' abilities or prevalence
in this particular domain.

\hypertarget{references}{%
\subsection*{References}\label{references}}
\addcontentsline{toc}{subsection}{References}

\hypertarget{refs}{}
\leavevmode\hypertarget{ref-aslanidis_populism_2018}{}%
Aslanidis, Paris. 2018. ``Populism as a Collective Action Master Frame
for Transnational Mobilization.'' \emph{Sociological Forum} 33 (2):
443--64. \url{https://doi.org/10.1111/socf.12424}.

\leavevmode\hypertarget{ref-bergman_quantitative_2018}{}%
Bergman, Matthew E. 2018. ``Quantitative Measures of Populism: A
Survey.'' SSRN Scholarly Paper ID 3175536. Rochester, NY: Social Science
Research Network. \url{https://papers.ssrn.com/abstract=3175536}.

\leavevmode\hypertarget{ref-carpenter_multilevel_2008}{}%
Carpenter, Bob. 2008. ``Multilevel Bayesian Models of Categorical Data
Annotation.'' Unpublished manuscript. unpublished manuscript.
\url{http://citeseerx.ist.psu.edu/viewdoc/download?doi=10.1.1.174.1374\&rep=rep1\&type=pdf}.

\leavevmode\hypertarget{ref-dai_measuring_2019}{}%
Dai, Yaoyao. 2019. ``Measuring Populism in Context: A Supervised
Approach with Word Embedding Models.''

\leavevmode\hypertarget{ref-dawid_maximum_1979}{}%
Dawid, Alexander Philip, and Allan M Skene. 1979. ``Maximum Likelihood
Estimation of Observer Error-Rates Using the EM Algorithm.''
\emph{Applied Statistics} 28 (1): 20--28.
\url{https://doi.org/10.2307/2346806}.

\leavevmode\hypertarget{ref-ernst_extreme_2017}{}%
Ernst, Nicole, Sven Engesser, Florin Büchel, Sina Blassnig, and Frank
Esser. 2017. ``Extreme Parties and Populism: An Analysis of Facebook and
Twitter Across Six Countries.'' \emph{Information, Communication \&
Society} 20 (9): 1347--64.
\url{https://doi.org/10.1080/1369118X.2017.1329333}.

\leavevmode\hypertarget{ref-guan_who_2017}{}%
Guan, Melody Y, Varun Gulshan, Andrew M Dai, and Geoffrey E Hinton.
2017. ``Who Said What: Modeling Individual Labelers Improves
Classification.'' \emph{arXiv Preprint arXiv:1703.08774}.

\leavevmode\hypertarget{ref-hawkins_is_2009}{}%
Hawkins, Kirk A. 2009. ``Is Chávez Populist?: Measuring Populist
Discourse in Comparative Perspective.'' \emph{Comparative Political
Studies} 42 (8): 1040--67.
\url{https://doi.org/10.1177/0010414009331721}.

\leavevmode\hypertarget{ref-hawkins_ideational_2018-1}{}%
Hawkins, Kirk A., Ryan E. Carlin, Levente Littvay, and Cristóbal Rovira
Kaltwasser. 2018. \emph{The Ideational Approach to Populism: Concept,
Theory, and Analysis}. Routledge.

\leavevmode\hypertarget{ref-hawkins_textual_2018}{}%
Hawkins, Kirk A, and Bruno Castanho Silva. 2018. ``Textual Analysis.''
In \emph{The Ideational Approach to Populism: Concept, Theory, and
Analysis}, edited by Kirk A. Hawkins, Ryan E. Carlin, Levente Littvay,
and Cristóbal Rovira Kaltwasser. Routledge.

\leavevmode\hypertarget{ref-hua_networked_2018}{}%
Hua, Whitney, Tarik Abou-Chadi, and Pablo Barberá. 2018. ``Networked
Populism: Characterizing the Public Rhetoric of Populist Parties in
Europe.'' In. Paper prepared for the 2018 MPSA Conference.

\leavevmode\hypertarget{ref-jagers_populism_2007}{}%
Jagers, Jan, and Stefaan Walgrave. 2007. ``Populism as Political
Communication Style: An Empirical Study of Political Parties' Discourse
in Belgium.'' \emph{European Journal of Political Research} 46 (3):
319--45. \url{https://doi.org/10.1111/j.1475-6765.2006.00690.x}.

\leavevmode\hypertarget{ref-krippendorff_content_2004}{}%
Krippendorff, Klaus. 2004. \emph{Content Analysis: An Introduction to
Its Methodology}. 2nd ed. Thousand Oaks, Calif: Sage.

\leavevmode\hypertarget{ref-march_textual_2018}{}%
March, Luke. 2018. ``Textual Analysis.'' In \emph{The Ideational
Approach to Populism: Concept, Theory, and Analysis}, edited by Kirk A.
Hawkins, Ryan E. Carlin, Levente Littvay, and Cristóbal Rovira
Kaltwasser. Routledge.

\leavevmode\hypertarget{ref-oliver_rise_2016}{}%
Oliver, J Eric, and Wendy M Rahn. 2016. ``Rise of the Trumpenvolk:
Populism in the 2016 Election.'' \emph{The ANNALS of the American
Academy of Political and Social Science} 667 (1): 189--206.

\leavevmode\hypertarget{ref-passonneau_benefits_2014}{}%
Passonneau, Rebecca J, and Bob Carpenter. 2014. ``The Benefits of a
Model of Annotation.'' \emph{Transactions of the Association for
Computational Linguistics} 2: 311--26.

\leavevmode\hypertarget{ref-pauwels_measuring_2011-1}{}%
Pauwels, Teun. 2011. ``Measuring Populism: A Quantitative Text Analysis
of Party Literature in Belgium.'' \emph{Journal of Elections, Public
Opinion \& Parties} 21 (1): 97--119.
\url{https://doi.org/10.1080/17457289.2011.539483}.

\leavevmode\hypertarget{ref-pauwels_measuring_2017}{}%
---------. 2017. ``Measuring Populism: A Review of Current Approaches.''
In \emph{Political Populism: A Handbook}, edited by Reinhard C.
Heinisch, Christina Holtz-Bacha, and Oscar Mazzoleni, 123--36. Nomos.

\leavevmode\hypertarget{ref-polk_explaining_2017}{}%
Polk, Jonathan, Jan Rovny, Ryan Bakker, Erica Edwards, Liesbet Hooghe,
Seth Jolly, Jelle Koedam, et al. 2017. ``Explaining the Salience of
Anti-Elitism and Reducing Political Corruption for Political Parties in
Europe with the 2014 Chapel Hill Expert Survey Data.'' \emph{Research \&
Politics} 4 (1): 1--9. \url{https://doi.org/10.1177/2053168016686915}.

\leavevmode\hypertarget{ref-rooduijn_populist_2014}{}%
Rooduijn, Matthijs, Sarah L de Lange, and Wouter van der Brug. 2014. ``A
Populist Zeitgeist? Programmatic Contagion by Populist Parties in
Western Europe.'' \emph{Party Politics} 20 (4): 563--75.
\url{https://doi.org/10.1177/1354068811436065}.

\leavevmode\hypertarget{ref-rooduijn_measuring_2011}{}%
Rooduijn, Matthijs, and Teun Pauwels. 2011. ``Measuring Populism:
Comparing Two Methods of Content Analysis.'' \emph{West European
Politics} 34 (6): 1272--83.
\url{https://doi.org/10.1080/01402382.2011.616665}.


\end{document}
